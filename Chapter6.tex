\chapter{Source Code Access}

The operating system realized is open source and is published at the address:

\hfill{}\url{https://github.com/t-crest/ospat}\hfill{}

The directory structure of the published project is the following:

\begin{verbatim}
.
|-- elf2uart
|   `-- src
|-- examples
|   `-- arinc653-1event-O1
|-- kernel
|   |-- arch
|   |-- core
|   |-- include
|   `-- middleware
|-- libpok
|   |-- arch
|   |-- arinc653
|   |-- core
|   |-- include
|   `-- middleware
`-- misc
    |-- ldscripts
    `-- mk
\end{verbatim}

\begin{itemize}
	\item \texttt{elf2uart}: contains the source code for the program streaming Patmos ELFs to the UART (see Section \ref{sec:elf2uart})
	\item \texttt{example}: contains source code and makefiles for a sample application
	\item \texttt{kernel}: contains source code of the OS kernel, \texttt{arch}, \texttt{core} and \texttt{middleware} contain the corresponding OS layers while \texttt{include} directory contains header files
	\item \texttt{libpok}: holds the OS library's source code
	\item \texttt{misc}: contains linker scripts (\texttt{ldscripts}) and make rules (\texttt{mk})
\end{itemize}

The Patmos simulator's code can be found at the address:

\hfill{}\url{https://github.com/t-crest/patmos/tree/master/simulator}\hfill{}

\section{Running an Example}

As stated before, the \texttt{example} directory contains the source code of a sample application. The example code is structured as follows:

\begin{verbatim}
.
|-- cpu
|   |-- kernel
|   |   |-- deployment.h
|   |   `-- Makefile
|   |-- Makefile
|   `-- part1
|       |-- activity.c
|       |-- activity.h
|       |-- deployment.h
|       |-- main.c
|       `-- Makefile
`-- Makefile
\end{verbatim}

The \texttt{makefile} in the root directory allows to build both the kernel and the partitions (in this example there is only one partition). The \texttt{kernel} directory contains a \texttt{makefile} to build the kernel and \texttt{deployment.h} file contains pre-compilation directives used to configure kernel's compilation. The \texttt{part1} contains the partition's code with \texttt{main.c} defining the partition's entry point and \texttt{activity.c} and \texttt{activity.h} contain partition's threads definition.\\

To run the sample application on the Patmos simulator, the script in listing \ref{lst:deploy sample application} can be run in the OS root directory.

\begin{lstlisting}[language=bash, caption=Bash script to run the sample application, label=lst:deploy sample application]
export ARCH=patmos
export POK_PATH=`pwd`

cd ./examples/arinc653-1event-O1/generated-code
make

cd $POK_PATH/elf2uart/
mkdir build
cd build
cmake ..
make

cd $POK_PATH
mkdir deploy

mv examples/arinc653-1event-O1/generated-code/cpu/pok.elf deploy/kernel.elf
mv examples/arinc653-1event-O1/generated-code/cpu/part1/part1.elf deploy/part1.elf
mv elf2uart/build/elf2uart deploy/elf2uart

cd deploy
touch deploy.uart
./elf2uart part1.elf --output=deploy.uart

pasim --in=deploy.uart --interrupt=1 kernel.elf
\end{lstlisting}

\setlength{\aboverulesep}{0pt}
\setlength{\belowrulesep}{0pt}
\setlength{\extrarowheight}{.75ex}
\newcolumntype{g}{>{\columncolor{light-gray}}l}
\begin{longtable}{g p{9cm}}
\toprule
\textbf{Lines 1-2} & Defines variables used in the build chain: path to the OS and target architecture\\
\midrule
\textbf{Lines 4-5} & Moves to the example's directory and starts the compilation\\
\midrule
\textbf{Lines 7-11} & Moves to the \texttt{elf2uart} directory and compiles the \texttt{elf2uart} utility\\
\midrule
\textbf{Lines 13-14} & Moves back to the root directory and creates the \texttt{deploy} directory to hold the example executables\\
\midrule
\textbf{Lines 16-18} & Moves kernel, partition and \texttt{elf2uart} executables to the \texttt{deploy} directory\\ 
\midrule
\textbf{Lines 20-22} & Moves to the \texttt{deploy} directory, creates a file to simulate the UART and streams the partition executable to the UART using \texttt{elf2uart}\\
\midrule
\textbf{Lines 24} & Runs the example\\
\bottomrule
\end{longtable}

