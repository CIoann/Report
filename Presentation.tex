\documentclass[notheorems]{beamer}
\usepackage{color}
\usepackage{amsmath}
\usepackage{eurosym}
\usepackage{booktabs}
\usepackage{tabularx}
\usepackage{rotating}
\usepackage{microtype}
\usepackage{indentfirst}
\usepackage{verbatim}
\usepackage{graphicx}
\usepackage{lmodern}
\usepackage[T1]{fontenc}
\usepackage[utf8x]{inputenc}
\usepackage{varwidth}
\usepackage{upgreek}
\usepackage{url}
\usepackage{caption}
\usepackage{subcaption}
\usepackage{listings}

\AtBeginSection{\frame{\sectionpage}}

\DeclareCaptionFont{white}{\color{white}}
\DeclareCaptionFormat{listing}{\colorbox{gray}{\parbox{\textwidth}{#1#2#3}}}
\captionsetup[lstlisting]{format=listing,labelfont=white,textfont=white}
\lstset{ %
  basicstyle=\footnotesize,           % the size of the fonts that are used for the code
  numbers=left,                   % where to put the line-numbers
  numberstyle=\tiny\color{gray},  % the style that is used for the line-numbers
  stepnumber=1,                   % the step between two line-numbers. If it's 1, each line 
                                  % will be numbered
  numbersep=5pt,                  % how far the line-numbers are from the code
  backgroundcolor=\color{white},      % choose the background color. You must add \usepackage{color}
  showspaces=false,               % show spaces adding particular underscores
  showstringspaces=false,         % underline spaces within strings
  showtabs=false,                 % show tabs within strings adding particular underscores
  %frame=single,                   % adds a frame around the code
  rulecolor=\color{black},        % if not set, the frame-color may be changed on line-breaks within not-black text (e.g. commens (green here))
  tabsize=2,                      % sets default tabsize to 2 spaces
  captionpos=t,                   % sets the caption-position to bottom
  breaklines=true,                % sets automatic line breaking
  breakatwhitespace=false,        % sets if automatic breaks should only happen at whitespace
  title=\lstname,                   % show the filename of files included with \lstinputlisting;
}

%\usepackage[a4paper,margin=1cm,landscape]{geometry}

\setbeamertemplate{frametitle continuation}{(\insertcontinuationcount)}
\setbeamertemplate{theorem}[ams style]
\setbeamertemplate{theorems}[numbered]
\setbeamertemplate{caption}[numbered]
\newtheorem{rules}{Rule}
\newtheorem{theorem}{Theorem}
\newtheorem{definition}{Definition}
\newtheorem{lemma}{Lemma}

\newcounter{bibcounter}
\newcommand{\biblabel}[1]{\refstepcounter{bibcounter} \label{#1}}

\setbeamerfont{caption}{size=\scriptsize}
\title[]{A Time-Composable Operating System for the Patmos Processor}
\author[Marco Ziccardi]{Marco Ziccardi \\
\texttt{marco.ziccard@gmail.com}}
\date[\today]{\today}
\institute[]{\includegraphics[scale=.7]{figures/DTU-logo.pdf}}
\usetheme{CambridgeUS}
\setbeamercovered{dynamic}

\setbeamercolor{item projected}{fg=white,bg=red}

\begin{document}

\begin{frame}
\maketitle
\end{frame}
\begin{frame}
\frametitle{Table of contents}
\tableofcontents
\end{frame}

\section{Introduction}

\begin{comment}
\begin{frame}
\frametitle{Real-time Systems}
\begin{block}{}
A Real-Time System (RTS) receives inputs and sends outputs to the hardware, working under strict constraints on its response time~\cite{Ben:2006:PCD:1406210}. A real-time system is made of several real-time applications.
\end{block}
\begin{itemize}
    \item A real-time application can also be viewed as a \textit{taskset} $\tau$, a static set of tasks~\cite{Davis:2011:SHR:1978802.1978814}
    \item Tasks and therefore real-time systems can be classified according to the consequences of a deadline miss~\cite{mall2009real} into:
	\begin{itemize}
		\item \textit{hard}: the task has to produce a result within its deadline otherwise the whole system is considered to have failed	
		\item \textit{firm}: the task is required to produce a result within its deadline. Eventual late results are discarded
		\item \textit{soft}: the task has an associated deadline which, however, is not absolute but expressed as an average response time required
	\end{itemize}
\end{itemize}
\end{frame}

\begin{frame}
\frametitle{Real-time Operating Systems}
In order to handle real-time systems complexity Real-time Operating Systems (RTOS) are used. RTOSs have to consider time as a key parameter~\cite{Tanenbaum:2007:MOS:1410217}.

Main features are:
\begin{itemize}
	\item Time sharing, several scheduling algorithms:
		\begin{itemize}
			\item A widely adopted tecnique is \textit{partitioning}
		\end{itemize}
	\item (Virtual) Memory allocation
	\item Inter-task communication
\end{itemize}

\end{frame}

\begin{frame}
\frametitle{Partitioning}
\begin{itemize}
	\item \textit{Static} \textit{non work-conserving} schedulers such as Time Division Multiplexing (TDM) assign applications to the processor
	\item No runtime decisions to schedule applications means time isolation or \textit{partitioning} between applications
	\item Two level scheduling: \textit{inter} and \textit{intra-application} schedulers
	\item Preemptive \textit{intra-application} schedulers reach better processor utilization than non preemptive ones
		\begin{itemize}
			\item At the cost of more overhead
		\end{itemize}
	\item ARINC653~\cite{ARINC653:2003} is a standard specification for time and space partitioning in safety-critical avionics real-time operating systems
		\begin{itemize}
			\item Time is divided in a number of slots called partitions
			\item Each partition is assigned an application
			\item In each partition a second-level scheduler can be defined
		\end{itemize}
\end{itemize}
		
\end{frame}
\end{comment}

\begin{comment}
\begin{frame}
\frametitle{Some partitioned RTOSs (1)}
\begin{itemize}
	\item \textbf{PikeOS}
		\begin{itemize}
			\item Commercial
			\item Supports ARINC653 standard
			\item Two level scheduling based on virtual machines
		\end{itemize}
	\item \textbf{INTEGRITY}
		\begin{itemize}
			\item Commercial
			\item Partitioning RTOS
			\item Uses the MMU for high performance memory protection
		\end{itemize}
	\item \textbf{LynxOS-178}
		\begin{itemize}
			\item Commercial
			\item Partitioning RTOS
			\item Uses the MMU for high performance memory protection
		\end{itemize}
\end{itemize}
\end{frame}

\begin{frame}
\frametitle{Some partitioned RTOSs (2)}
\begin{itemize}
	\item \textbf{LynxOS-178}
		\begin{itemize}
			\item Commercial
			\item Time and space-partitioning through virtual machines
			\item Supports the ARINC653 standard
		\end{itemize}
	\item \textbf{CompOS}
		\begin{itemize}
			\item Composable OS
			\item Targes multiprocessor, each processor executes an independent instance of CompOSe
			\item Communication API to preserve isolation
		\end{itemize}
\end{itemize}
\end{frame}
\end{comment}

\begin{frame}
\frametitle{The T-CREST Project}

		\begin{figure}
			\includegraphics[scale=.4]{figures/TCREST.jpg}
		\end{figure}

\begin{itemize}
	\item Aims to create a time-predictable chip multi-processor
	\item Develops processor (Patmos~\cite{patmos:ppes2011}), memories and the interconnect
	\item Makes use of time-predictable caches to address the increased needs of processing power
\end{itemize}
\end{frame}

\section{The Patmos Processor}
\begin{frame}
\frametitle{Patmos}

\begin{itemize}
	\item Reduced instruction set architecture (RISC) processor
	\item Aims to reduce the complexity of WCET analysis
	\item Pipeline divided in 5 stages: \texttt{FE}, \texttt{DEC}, \texttt{EX}, \texttt{MEM} and \texttt{WB}
	\item Instruction Set Architecture (ISA) includes:
		\begin{itemize}
			\item Integer and binary arithmetic
			\item Loads/Stores to local and global memory for different types of data
			%\item Move instructions between different types of registers
			\item Branches, calls and return 
			\item Stack cache management instructions
		\end{itemize}
\end{itemize}
\end{frame}

\begin{frame}
\frametitle{Patmos Memories (1)}
Patmos uses several local memories:
\begin{itemize}
	\item Method cache~\cite{Schoeberl04atime}
		\begin{itemize}
			\item Idea: functions are usually short
			\item Cache divided in blocks able to hold whole functions
			\item Cache misses only on function calls
		\end{itemize}
	\item Data cache
		\begin{itemize}
			\item Write-through policy
			\item Higher number of memory accesses
			\item More analyzable WCET
		\end{itemize}
\end{itemize}
\end{frame}

\begin{frame}
\frametitle{Patmos Memories (2)}
\begin{itemize}
	\item Stack cache~\cite{patmos:stack:seus}
		\begin{itemize}
			\item Holds stack allocated values
			\item Managed by the compiler through three instructions:
				\begin{itemize}
					\item \texttt{reserve}: reserves space on the stack cache, may spill data
					\item \texttt{ensure}: ensures that caller data is in the stack cache, may spill data
					\item \texttt{free}: frees reserved space on the stack
				\end{itemize}
		\end{itemize}
	\item Data scratchpad
		\begin{itemize}
			\item Local memory
			\item Accesses do not stall the pipeline
		\end{itemize}
\end{itemize}
\end{frame}

\begin{frame}
\frametitle{Patmos Registers}
\begin{itemize}
	\item \texttt{R}: 32 general-purpose registers (32 bit), \texttt{r0} is set to \texttt{0} and read-only
	\item \texttt{S}: 16 special-purpose registers (32 bit)
	\item \texttt{P}: 8 predicate registers (1 bit), by convention \texttt{p0} is set to true (\texttt{1})
\end{itemize}
Compiler convention define how to use \texttt{R}:
\begin{itemize}
	\item \texttt{r1} and \texttt{r2}: function's result (up to 64 bits)
	\item \texttt{r3-r8}: function's arguments
	\item \texttt{r27}: temporary register
	\item \texttt{r28}: frame pointer
	\item \texttt{r29}: shadow stack pointer
	\item \texttt{r30}: address of the return function (function base)
	\item \texttt{r31}: offset of the return instruction from \texttt{r30} (function offset)
\end{itemize}
\end{frame}

\section{The TiCOS Operating System}
\begin{frame}
\frametitle{TiCOS}
\begin{block}{}
TiCOS~\cite{baldovin2012time} is a time-composable real-time operating system
\end{block}
\begin{itemize}
	\item Hard real-time
	\item Light weight
	\item Supports ARINC653
		\begin{itemize}
			\item Software specification for time and space partitioning
			\item Used in safety-critical avionics real-time operating systems
		\end{itemize}
	\item Originally targeted the PPC architecture
	\item Based on POK~\cite{delange11osadlpok}
\end{itemize}
\end{frame}

\begin{frame}
\frametitle{TiCOS Architecture}

	\begin{figure}[!ht]
		\begin{center}
		\makebox[\linewidth]{\includegraphics[scale=0.7]{figures/OS_structure.pdf}}
		\end{center}
		%\caption{High level structure of the OS (from~\cite{baldovin2012time})}
		\label{OS_structure}
	\end{figure}

\end{frame}

\begin{frame}
\frametitle{TiCOS Kernel Layer}
\begin{itemize}
	\item Divided in \textit{arch component} and \textit{core component}
	\item Provides several services
		\begin{itemize}
			\item Partitions support
				\begin{itemize}
					\item Each partition is a real-time application
					\item Partitions are time and space isolated
				\end{itemize}
			\item Two level scheduling
				\begin{itemize}
					\item Cyclic scheduling of partitions (a time slot is assigned to each partition)
					\item Constant-time (O(1)) fixed-priority scheduler between tasks in a partition
					\item Run-to-completion semantics through delayed preemption
				\end{itemize}
			\item Lock objects management
			\item Inter-partition communication management
				\begin{itemize}
					\item No shared memory
					\item Using ARINC ports
				\end{itemize}
		\end{itemize}
\end{itemize}
\end{frame}

\begin{comment}
\begin{frame}
\frametitle{ARINC653 Entities}
\begin{itemize}
	\item Events
	\item Semaphores
	\item Blackboards
	\item Buffers
	\item Sampling Ports
	\item Queuing Ports
\end{itemize}
\end{frame}
\end{comment}

\begin{frame}
\frametitle{TiCOS Library}
Is made of three parts:
\begin{itemize}
	\item Core library
		\begin{itemize}
			\item Allows to access to core functionalities (like system calls)
			\item Used by ARINC library
			\item Non-portable applications
		\end{itemize}
	\item Middleware library
		\begin{itemize}
			\item Allows to access some of the ARINC services
			\item Not part of the ARINC specification
			\item Non-portable applications
		\end{itemize}
	\item ARINC library
		\begin{itemize}
			\item Allows to create and manage ARINC structures
			\item Follows ARINC standard (portable applications)
		\end{itemize}
\end{itemize}
\end{frame}

\section{Processor Extensions}
\begin{frame}
\frametitle{RTC and Interrupts}
\begin{block}{}
Interrupts are used to stop the current flow of control jumping to a dedicated Interrupt Service Routine (ISR) in order to react to external events~\cite{t-crest:d5.1}
\end{block}
\begin{itemize}
	\item Patmos extended with a RTC
	\item RTC registers mapped to the local memory
		\begin{itemize}
			\item Clock cycles counter (64 bits)
			\item Time in microseconds (64 bits)
			\item Interrupt interval
			\item ISR address
		\end{itemize}
	\item Interrupt interval decremented every cycle
	\item When interrupt interval reaches 0 an interrupt is fired, program counter is stored in \texttt{s9} and ISR is called
	\item Memory mapped RTC implemented in the Patmos simulator
\end{itemize}
\end{frame}

\begin{frame}
\frametitle{Stack Cache Manipulation}
\begin{itemize}
	\item Stack cache may not be consistent with memory stack
	\item A thread's context has to contain stack cache status
	\item Stack cache has to be spilled to main main 
	\item Original instructions (\texttt{reserve}, \texttt{ensure} and \texttt{free}) take immediate parameters
	\item \texttt{reserve}, \texttt{ensure} and \texttt{free} cannot spill and restore a dynamic amount of stack cache
	\item Two new instructions are introduced:
		\begin{itemize}
			\item \texttt{spill} (with register parameter)
			\item \texttt{ensure} (with register parameter)
		\end{itemize}
\end{itemize}
\end{frame}

\section{TiCOS Extensions}
\begin{frame}
\frametitle{Porting Technique}
Changes to the TiCOS operating system started from the architecture-dependent component up to the OS library, according to an incremental development technique:
\begin{itemize}
	\item Kernel's arch component
	\item Kernel's core component
	\item OS library
\end{itemize}
\end{frame}

\subsection{Kernel's Arch Component}
\begin{frame}[fragile]
\frametitle{Clock}
\begin{itemize}
	\item Ways to access memory mapped RTC registers were needed
	\item Local memory load/store instructions are used
	\item Header file \texttt{rtc.h} containing access macros
\end{itemize}

\hspace{0.5cm}~\begin{minipage}[c]{.9\textwidth}
\begin{lstlisting}[language=C, width=.9\textwidth, caption=\texttt{rtc.h} header file]
#define _IODEV __attribute__((address_space(1)))
typedef _IODEV unsigned int volatile * const _iodev_ptr_t;

//...

#define __PATMOS_RTC_ISR_ADDR (0xF0000314)
#define __PATMOS_RTC_WR_ISR(address) 
    *((_iodev_ptr_t)__PATMOS_RTC_ISR_ADDR) = address;
\end{lstlisting}
\end{minipage}
\end{frame}

\begin{frame}
\frametitle{Thread's Context}
Patmos thread's context has to include:
\begin{itemize}
	\item Stack cache
		\begin{itemize}
			\item On a context switch the stack cache is spilled to the thread's stack memory area
		\end{itemize}
	\item General-purpose caller-saved (scratch) registers (\texttt{r1-r19})
	\item General-purpose callee-saved registers (\texttt{r20-r31})
	\item Special-purpose registers (\texttt{s0-s15})
	\item Amount of thread's stack stored in the cache (\texttt{ssize})
		\begin{itemize}
			\item Needed to restore the stack cache status
		\end{itemize}
\end{itemize}
\end{frame}

\begin{frame}
\frametitle{Memory Management}

\begin{minipage}[c]{.35\textwidth}
	\begin{figure}[!ht]
		\begin{center}
		\makebox[\linewidth]{\includegraphics[scale=0.48]{figures/memory_layout_slide.pdf}}
		\end{center}
		\caption{Memory layout created by the operating system}
		\label{fig:memory layout}
	\end{figure}
\end{minipage}~
\begin{minipage}[c]{.6\textwidth}
\begin{itemize}
	\item Patmos uses a cached stack and a shadow stack dedicated to hold aliased data
	\item Several memory areas have to be allocated
		\begin{itemize}
			\item A cached stack and a shadow stack for the kernel
			\item A cached stack and a shadow stack for each partition (\textit{main thread})
			\item A cached stack and a shadow stack for each partition's task
		\end{itemize}
	\item Some memory is needed to hold thread's contexts
	\item The memory layout is shown in figure \ref{fig:memory layout}
\end{itemize}
\end{minipage}
\end{frame}

\subsection{Kernel's Core Component}
\begin{frame}
\frametitle{Bootloader}

\begin{minipage}[c]{.6\textwidth}
\begin{itemize}
	\item Originally partition were put in the kernel's binary
	\item A more flexible bootloader has been developed
		\begin{itemize}
			\item Partitions are loaded from the \texttt{UART}
			\item Partition's stream is expected to be formatted like in figure \ref{fig:partition stream format}
			\item \texttt{elf2uart} utility to read an executable and stream it to the \texttt{UART}
		\end{itemize}
\end{itemize}
\end{minipage}~
\begin{minipage}[c]{.35\textwidth}
	\begin{figure}[!ht]
		\begin{center}
		\makebox[\linewidth]{\includegraphics[scale=0.7]{figures/partition_stream_format.pdf}}
		\end{center}
		\caption{Format of the partition's stream expected by the operating system on the \texttt{UART}}
		\label{fig:partition stream format}
	\end{figure}
\end{minipage}
\end{frame}

\subsection{OS Library}
\begin{frame}
\frametitle{System Calls Implementation}
\begin{itemize}
	\item Patmos has no system call instruction
	\item No interrupt associated to system call
	\item \texttt{system\_call} function is placed in a known memory location
	\item To perform a system call the code jumps to \texttt{system\_call} address
		\begin{itemize}
			\item The required system call is identified
			\item The proper kernel function is called
		\end{itemize}
\end{itemize}
\end{frame}

\subsection{Context Switch}
\begin{frame}
\frametitle{Context Switch}
Two types of context switch:
\begin{itemize}
	\item Interrupt-driven context switch
		\begin{itemize}
			\item Interrupt interval set to expire at specified points
			\item Used at the end of a time slot
			\item \texttt{PC} is stored in \texttt{s9}
		\end{itemize}
	\item Explicit context switch
		\begin{itemize}
			\item Caused by the run-to-completion semantics
			\item When a periodic task ends or an event becomes ``up'', unlocking a sporadic task
			\item Return address computed as \texttt{r30+r31}
		\end{itemize}
	\item \texttt{current\_context} is a global pointer to the current executing taks's context
\end{itemize}
\end{frame}

\begin{frame}
\frametitle{Interrupt-driven Context Switch}

	\begin{figure}[!ht]
		\begin{center}
		\makebox[\linewidth]{\includegraphics[scale=0.75]{figures/context_switch_flow.pdf}}
		\end{center}
		%\caption{Flow of function calls needed to perform a context switch}
		\label{fig:context switch flow}
	\end{figure}

\end{frame}

\begin{frame}
\frametitle{Explicit Context Switch}
\begin{itemize}
	\item Assembly function \texttt{context\_switch(uint32\_t context)} explicitly called
	\item \texttt{context} holds the address where to store the current context
	\item \texttt{current\_context} holds the context to restore
	\item Context switch is handled through two functions:
		\begin{itemize}
			\item \texttt{context\_switch}: stores the context to \texttt{context}
			\item \texttt{restore\_context}
		\end{itemize}
	\item When \texttt{context\_switch} is called the next taks has already been choosen
\end{itemize}
\end{frame}

\section{Conclusions}
\begin{frame}
\frametitle{Contributions}
\begin{itemize}
	\item Porting TiCOS operating system to the Patmos processor
	\item Identifying needed extensions to the Patmos processor
	\item Adding an OS to the T-CREST time-predictable platform made of processor and compiler
	\item Identifying and solving bugs in the Patmos simulator and compiler
	\item Creating a solid starting point for future work (e.g. porting another OS)
\end{itemize}
\end{frame}

\begin{frame}
\frametitle{Future Work}
\begin{itemize}
	\item Extend the OS to multi-processors
	\begin{itemize}
		\item Starting from independent instances of the OS on each processor
	\end{itemize}
	\item Test the OS on the hardware (as soon as it's ready)
	\item Measure OS performances on the hardware
	\item Compare functionalities and performances with other OS (as soon as they're ready)
\end{itemize}
\end{frame}

\begin{frame}[allowframebreaks]
		\tiny
        \frametitle{References}
		\bibliographystyle{plain}
		\bibliography{References}
\end{frame}



\end{document}

