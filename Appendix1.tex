\chapter{Software Simulator of Patmos}\label{Software simulator of Patmos}

The Patmos simulator~\cite{t-crest:d2.1} is a C++ implementation made of classes and interfaces representing processor's component and is designed to be modular and extensible. A class diagram for the Patmos simulator can be seen in figure \ref{fig:patmos_simulator}.

	\begin{figure}[!ht]
		\begin{center}
		\makebox[\linewidth]{\includegraphics[scale=0.8]{figures/patmos_simulator.pdf}}
		\end{center}
		\caption{Class diagram for the Patmos software simulator}
		\label{fig:patmos_simulator}
	\end{figure}


\texttt{simulator\_t} is the main class of the simulator and controls the whole simulation. This main class must be connected to the components of the processor and so we have:

\begin{itemize}
	\item \texttt{decoder\_t}: is the component responsible for the instruction decode. The decoder translates ISA instruction and translates them to simulator instructions.
	\item \texttt{memory\_t}: abstract class for the memory, specifies a general behavior a memory should expose such as read and write operations. This abstract class is implemented by two other classes: \texttt{ideal\_memory\_t} and \texttt{fixed\_delay\_memory\_t}. The simulator uses this class to reference both processor's global and local memory.
	\item \texttt{register\_file\_t}: template class which offers methods for getting and setting each register in a group of registers. Allows to implement predicate register (\texttt{PGR}), general purpose registers (\texttt{GPR}) e special purpose registers (\texttt{SPR}).
	\item \texttt{data\_cache\_t}: another abstract class extending \texttt{memory\_t}, is the root of a hierarchy made of two other subclasses: \texttt{ideal\_data\_cache\_t} and \texttt{lru\_data\_cache\_t}. Both those classes implement a type of data cache so both wrap an other memory object to cache.
	\item \texttt{stack\_cache\_t}: another abstract class extending \texttt{memory\_t}, is the root of a hierarchy made of two other subclasses: \texttt{ideal\_stack\_cache\_t} and \texttt{block\_stack\_cache\_t}. Both those classes implement a type of stack cache and both wrap the global memory.
\end{itemize}

\texttt{instruction\_t} is another important class which wraps the concept of a processor instruction. The class offers a method for each pipeline stage (\IF, \DR, \EX, \MW) which actually executes the corresponding pipeline stage. \texttt{instruction\_data\_t} class wraps an instruction and its operands. The simulator pipeline is nothing more than a bidimensional array of \texttt{NUM\_STAGES x NUM\_SLOTS} of \texttt{instruction\allowbreak\_data\_t} objects, where \texttt{NUM\_STAGES} is the number of processor pipeline's stages and \texttt{NUM\_SLOTS} is the number of slots in each bundle.\\

The simulation is handled by a loop in the \texttt{simulator\_t} class. At each iteration a CPU cycle is executed and the following operations are performed:

\begin{enumerate}
	\item The next \texttt{NUM\_SLOTS} is decoded and the program counter is incremented (only if the \texttt{IF} pipeline stage is not stalled).
	\item For each instruction in the pipeline the method corresponding to the stage where the instruction is placed is executed.
	\item Each instruction is advanced to the next pipeline stage. In case some stage is stalled only a subset of those instructions are advanced.
	\item Memory and cache state is advanced.
\end{enumerate}

\section{Instruction Simulation}

The class \texttt{instruction\_t} is the root of a hierarchy of classes implementing each a processor instruction, this hierarchy is shown in figure \ref{instruction_hierarchy} and allows to reuse implementations common to group of instructions.

	\begin{figure}[!ht]
		\begin{center}
		\makebox[\linewidth]{\includegraphics[scale=0.6]{figures/instruction_hierarchy.pdf}}
		\end{center}
		\caption{Class diagram for simulator's instructions hierarchy}
		\label{instruction_hierarchy}
	\end{figure}

\section{Memory and Cache Simulation}

Memories, caches and memory mapped devices can be accessed through abstract classes which isolate from their actual implementation. The abstract class \texttt{memory\_t} is extended by other three abstract classes: \texttt{data\_cache\_t}, \texttt{stack\_cache\_t} and \texttt{memory\_map\_t}.

	\begin{figure}[!ht]
		\begin{center}
		\makebox[\linewidth]{\includegraphics[scale=0.7]{figures/memory_hierarchy.pdf}}
		\end{center}
		\caption{Class diagram for simulator's memory hierarchy}
		\label{memory_hierarchy}
	\end{figure}

\texttt{memory\_map\_t} extends \texttt{memory\_t} and so it offers the same interface. \texttt{memory\_map\allowbreak \_t} wraps an object of the class \texttt{memory\_t} extends its base class with a collection of \texttt{mapped\_device\_t} and offers methods to add object of this type to that collection. The class \texttt{mapped\_device\_t} represents a portion of memory which corresponds to the content of the registers of a memory mapped IO device (for example an UART). When a read/write method is called for a specific address on an object of the class \texttt{memory\_map\_t} all the \texttt{mapped\_device\_t} are inspected to look for a device having a mapped register for the address, if an object is found the read/write is forwarded otherwise the wrapped \texttt{memory\_t} is accessed. In the case of Patmos simulator the \texttt{memory\_map\_t} object is wrapped around the local memory, so memory mapped devices can be accessed through local memory read/write instructions.

