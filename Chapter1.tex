\chapter{Introduction}

The process of software development doesn't feel any better than it did a generation ago. Yet researchers pointed out certain aspects of the problem still the workflow and the way a progrmamer might thing at a particular moments is still a mystery a black box 

understanding how they do it, and how they keep the dtructures, the strategies and how the tools are used is not an easy task. in order to improve the environments available we have although to understand these aspects.

ides aim to help the user however often they create a general chaos with all the navigation the studies of .. and the Cognition studies and models strategies top down bottom up, 

previous research of understanding the workflow of a developer is done by ... 

previous research of attempting to extracct user activities from IDEs ... 

In this chapter we introduce some key aspects of Real-Time Systems (RTS), starting from the computational model. We then discuss \textit{timing analysis} and Worst-Case Execution Time (WCET) calculation. After that we introduce \textit{time-composability} and how important that property is. In Section \ref{sec:RTOS} the role of a Real-Time Operating System (RTOS) is described and some examples are presented. In the end the T-CREST project and platform are briefly described.

\section{Integrated Development Environment}
	\todo{list available and discuss}
	\todo{paragraph why we choose eclipse}
	\todo{programmer workflow within IDE}
\section{The problem}
\section{The approach}
\section{Structure of thesis}
