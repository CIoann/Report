\chapter{Abstract}



Software developers interact with integrated development environments (IDEs) by issuing commands that execute various programming tools, from source code formatters to build tools. Several studies has shown that despite the usefulness of provided tools and capabilities to perform tasks such as: navigating among classes and methods, continuous compilation, code refactoring and integrated debugging, IDEs usually tend to overload developers resulting a chaotic state. In other words, commonly developers spend time shifting between several artifacts of the working environment in order to reach their goal, and this repeated process increases the complexity of identifying the relevant information to solve their assignment. 

For this reason, an assortment of usage metrics plugins to gather developers' activity have been developed in an attempt to understand a developers' workflow. Understanding the workflow, as well as the skillset and experience of a developer is the first step into improving their productivity and reducing the unneccessary complexity created within IDEs. 

In this work, we analyze and provide an overview of available existing plugins for recording a developer's interaction within an IDE and we further enhance a tool in order to capture, mine and analyze the process of software developement. We aim to contribute information for modelling an automatic activity detection tool capable of tracking the workflow of developers’ activity which potentially will be used as input in later studies for a reconfiguration program supporting the developer.

In more depth, this thesis will provide an overview of contextual factors and an analysis of how they can be achieved using already existing plugins that allow recording developers’ activity within Eclipse IDE. In addition, is expected to uncover and characterise developer’s workflow.
\todo{write}
%we also examine the link between programmer's interaction and some of the contextual factors of software development
%In this work, i'd like to propose a model to present the user interactions I gather from the IDE. in parallel with the process mining we gonna make

