\chapter{\gls{ide}}\label{cha:IDEs}

During programming tasks developers often interact with a variety of facilities and tools: browsers, \gls{ide}s, mail clients, virtual machines, etc. All these interactions are of high significance when it comes to understand the workflow of a developer. An \gls{ide} is very personal and therefore analysing developers' behavior is feasible. For this reason, the main focus of this chapter is the analysis of interactions happening inside of an \gls{ide}. 

Firstly an introduction to the history of \gls{ide} and a discussion on possible interactions with today's \gls{ide} follows.

\section{\gls{ide} history}

An \gls{ide} is a software application which aims to improve developers' productivity by facilitating application development. It consolidates the basic tools developers need to write and test software. Typically, an \gls{ide} consists of a source code editor, a compiler, a debugger and build automation tools. According to \todo{reference: https://www.veracode.com/security/integrated-development-environments} the idea behind \gls{ide} was realeased as Turbopascal which integrated an editor and a compile. However many believe Microsoft's Visual Basic (VB), launched in 1991 was the first real \gls{ide}
.

the first ide to come around and then how they evolved.


Today, according to the Top IDE index \todo{https://pypl.github.io/IDE.html}, a ranking created by analysing how often \gls{ide}s are searched on Google, the 3 most used tools employed to develop source code are Eclipse, Visual Studio, Android Studio. Although \gls{IDE}s are stricking through the top 3, it is noticable that a significant amount of developers prefer to use highly configurable text editors such as Vim and Sublime Text. For this thesis the decision to analyse and develop for Eclipse \gls{ide} was taken.

and the problem with ide's today

\section{Interactions within \gls{ide}}
\section{Summary}